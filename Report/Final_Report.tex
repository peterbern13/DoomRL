\def\year{2019}\relax
%File: formatting-instruction.tex
\documentclass[letterpaper]{article} %DO NOT CHANGE THIS
\usepackage{aaai19}  %Required
\usepackage{times}  %Required
\usepackage{helvet}  %Required
\usepackage{courier}  %Required
\usepackage{url}  %Required
\usepackage{graphicx}  %Required
\frenchspacing  %Required
\setlength{\pdfpagewidth}{8.5in}  %Required
\setlength{\pdfpageheight}{11in}  %Required
%PDF Info Is Required:
  \pdfinfo{
/Title (2019 Formatting Instructions for Authors Using LaTeX)
/Author (AAAI Press Staff)}
\setcounter{secnumdepth}{0}  
 \begin{document}
% The file aaai.sty is the style file for AAAI Press 
% proceedings, working notes, and technical reports.
%
\title{Fall 2018 CS7180 Final Project Report}
\author{AAAI Press\\
Peter Bernstein and Giorgio Severi
}
\maketitle
\begin{abstract}
AAAI creates proceedings, working notes, and technical reports directly from electronic source furnished by the authors. To ensure that all papers in the publication have a uniform appearance, authors must adhere to the following instructions. 
\end{abstract}

\section{Introduction}
 

%Doom, a classical First-Person Shooting (FPS) game created in 1993. It was designed to have a plot in which most of the components are sci-fi and horror: the player is a marine who is in the central point of an invasion of demons from hell. Our agent is going to play in only one scenario which is named Doom Defend Line (Figure 1). In this scenario, our agent is in a closed room and has to battle against several demons that can be classified in two types: demons throwing fire balls (they only move to the sides), and short-distance attack demons that slowly approach to our agent. Both of them respawn. The specific characteristics of the game are as follow. The objective on each episode is to kill as many monsters. For every monster killed, 1 point is received, and for every time our agent is killed, one point is subtracted. There are three possible actions for our agent: turn left, turn right and shoot. To complete the goal proposed in the simulator, the agent needs to reach 15 as the best 100-episode average reward. To get 15 points in one episode our agent either has to kill 16 monsters before being killed, or kill 15 monsters without being killed. In this work, we create a machine learning agent that play Doom Defend Line, which is a scenario from the classic game Doom. We hypothesize that by using two different RL methods separately we can create an agent that is competi-tive enough to score 15 points in average through 100 con-secutive episodes which is the goal proposed in the simula-tor website. 





\section{Background}
Reinforcement learning has been widely used to successfully solve a variety of games. Doom is a classic First-Person Shooter (FPS) game from 1993 created by Id Software. As one of the first FPS games to market, it is simple in its gameplay in comparison to anything modern, but is still significantly more complex than many other Atari games, such as Pong. In Doom, the user can move, attack, and pick up items and the game state may not be completely visible at all times and obstacles often partially obscure the display. These complexities make Doom an interesting study for reinforcement learning while in this project we focus on the simple case of the user encountering a single adversary in a match. 


\section{Related Work}



\section{Project Description}

\section{Experiments}

\section{Conclusion}




















\end{document}